\section{Project Insights}\label{sec:insights}

This study has also shown that a solution to the boot reliability problem need not be costly in terms of either time or effort.  Two flash storage devices are enough to provide for kernel integrity, and many off-the-shelf devices provide support for an extra storage device.  Power cycling flash memory on a daily basis is enough to provide almost three 9s of reliability and should only require a few minutes per day to perform the power cycle.

\subsection{Further Recommendations}
Based on other observations during this project and past experience, we would
like to make the following recommendations to the IlliniSat team.

\subsubsection{Reboot on panic} By default, when the Linux kernel fails, it
issues a ``kernel panic'' message and halts the system.  While IlliniSat's
watchdog should detect this scenario, the kernel has the ability to
automatically reboot on panic.  Enabling this can help ensure the system will
reboot in the event of a transient kernel error. The system will either reboot
faster than waiting for a watchdog timeout since the kernel will issue the
reboot almost immediately after failure or it will still reboot in the unlikely
event of a watchdog failure.

\subsubsection{OOM-killer} The Linux kernel that the IlliniSat team is using
has an Out-Of-Memory killer (OOM-killer) built into it. When the system is 
low on main memory, the OOM-killer will kill a process based on some heuristics.
However, this can have undesirable consequences if it kills certain processes, such
as the heart monitor or other mission critical components. We recommend that the
IlliniSat team uses the OOM-killer but that they either tune the OOM-killer such 
that processes are killed in an acceptable order or that the kernel be set
to panic on OOM~\cite{oracleoom}.

%OOM Merge anchor
\subsubsection{Regular Reboot} Our results from the M\"obius simulation model
in Table \ref{tab:reliability} support a regular reboot of the system. For a one year
mission, 1000 days without reboot is effectively the same as never rebooting. The
difference between this and rebooting every day for 2 and 3 devices increases the
reliability from 0.618 to 0.998 and 0.740 to 1.000 respectively. This is a change in
38\% and 26\% respectively which is a very large increase in system reliability. 
Even in thme case of a single device, the reliability increases by 7.3\% from 39.5\%
to 46.8\% Because of the increase in reliability with little to no additional overhead, 
we urge the team to force regular reboots of the system every day.
