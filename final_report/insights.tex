\section{Project Insights}\label{sec:insights}

This study has also shown that a solution to the boot reliability problem need not be costly in terms of either time or effort.  Two Flash storage devices are enough to provide for kernel integrity, and many off-the-shelf devices provide support for an extra storage device.  Power cycling Flash memory on a daily basis is enough to provide almost three 9s of reliability and should only require a few minutes per day to perform the power cycle.

\subsection{Further Recommendations}
Based on other observations during this project and past experience, we would
like to make the following recommendations to the IlliniSat team.

\subsubsection{Reboot on panic} By default, when the Linux kernel fails, it
issues a ``kernel panic'' message and halts the system.  While IlliniSat's
watchdog should detect this scenario, the kernel has the ability to
automatically reboot on panic.  Enabling this can help ensure the system will
reboot in the event of a transient kernel error. The system will either reboot
faster than waiting for a watchdog timeout since the kernel will issue the
reboot almost immediately after failure or it will still reboot in the unlikely
event of a watchdog failure.

\subsubsection{OOM-killer} When the system is low on main memory, the
Out-Of-Memory killer (OOM-killer) will kill a process based on some heuristics.
However, this can have undesirable consequences if it kills certain processes.
We recommend that IlliniSat either be mindful of this and tune the OOM-killer
such that processes are killed in an acceptable order or that the kernel be set
to panic on OOM~\cite{oracleoom}.
