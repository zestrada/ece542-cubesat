\section{Project Impact}\label{sec:impact}

The impacts of this project have both a specific and general component. Specifically, 
we will be able to provide analysis to the IlliniSat team. Since we designed the
M\"obius model to use paramaters that match the IlliniSat specification, we
can give the team a realistic probability of failure given different failure modes.
Also, we can suggest improvements to their system to greatly improve their
reliability and probability for a successful mission. In addition, all of the software
was designed to be compatible with the IlliniSat software stack. Our created software
avoided complex solutions to make it easily fit in with IlliniSat and with the team.

In a more general sense, all of what we have done can be easily tweaked to
fit any other similar CubeSat projects. The M\"obius model parameters default
to the IlliniSat specifications, but can be changed to account for any differences
between the hardware of a different mission. Likewise, the software can be
adapted to different projects. The U-boot modifications can be adapted to U-boot
procedures on other boards. The multiple kernel images can even be used in other
boot procedures, but the reliability analysis that we have given shows that adding
these copies on multiple devices can greatly increase reliability. As for the software
daemons, they are only dependent on the kernel running linux 2.6 or above so that
it has the inotify tools necessary to run.

Since CubeSats are becoming increasingly popular, providing a realistic model is
invaluable to any team that is starting out. This starting point can help teams plan
their missions, from the length of the mission all the way down to the hardware that
they use. The improvements to the IlliniSat mission can also give the teams insight
into simple reliability additions to their current or new systems that are low in cost
and also power consumption.