\section{Future Work}\label{sec:future_work}
\begin{itemize}
  \item {\bf Run IlliniSat code natively in QEMU.} It was useful to have QEMU
  and run our tests on ARM compiled code, but we were unable to use the exact
  same binaries (e.g. U-Boot) since the devices and memory maps in the
  versatilepb system are different from the MitySOM-335x.  If we were able to
  add the MitySOM-335x to QEMU, we could verify the entire software stack of
  IlliniSat.  This would be somewhat of a large effort, but very worthwhile as
  it would enable us to run different experiments - including much more thorough
  fault injection campaigns.  Furthermore, it would provide a very useful
  software development environment for the IlliniSat team.
  \item {\bf Model different hardware.}  While the M\"obius simulation used in this experiment was tuned to the board being used by the IlliniSat team, CubeSat is designed to be used on low-cost commodity hardware.  Studies such as \cite{Oldham2008TID} found a four-order-of-magnitude difference in flash memories of the same size from different manufacturers.  Our solution to the boot protection problem was designed to minimize the effects of SEFI failures because the IlliniSat hardware has a relatively high SEFI rate; we would most likely recommend different strategies for memories where the risk of SEU is non-negligable and the risk of SEFI is less significant.
  \item {\bf Model longer mission lifetimes.}  One of the major assumptions that guided the development of our fault model was the assumption that the IlliniSat mission had at most a 1 year lifetime.  This assumption simplified the model because the Total Ionizing Dose (TID) that the satellite received was small enough to have no effect during the course of the mission \cite{Likar2010Novel, Oldham2008TID}.  However, TID-related failure is a significant failure mode that could occur during longer missions.  By extending the model to support larger mission lifetimes, we would be able to better characterize reliability challenges for long-term CubeSat missions.
  \item {\bf Provide additional software assurance.} While this project
  primarily focused on providing fault-tolerance from hardware errors via
  software, we believe that a formal analysis of the IlliniSat software stack
  would also help to provide dependability.  As our failure model only considers
  data corruption, it is unable to protect against all application-level crashes
  or other errors that could exist in the IlliniSat software.  We chose not to
  work on this topic for our project because the software is currently being
  developed; however, we could certainly revisit it in the future.
\end{itemize}
