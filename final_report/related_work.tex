\section{Related Work}\label{sec:related_work}
There is already an existing body of work that deals with small satellites and
investigations of their reliability.  In \cite{odegaard2013error}, the author
discusses developing software systems to increase the reliability of a small satellite.
The use of a recovery mode, one of the systems discussed in
\cite{odegaard2013error}, is already present in IlliniSat.
\cite{odegaard2013error} also proposes solutions for protecting the satellite's
program code, although its implementation is unable to handle the case of a
corrupted kernel with no saved system checkpoints; our proposed approach stores
multiple copies of the kernel, so it is able to recover even if the kernel image
is completely corrupted.

There has also been significant work in using hardware-based reliability measures in small satellite designs.  \cite{toorian2008cubesat} mentions redundancy and watchdog timers as two solutions to ensure CubeSat reliability.  In \cite{passerone2008design}, the authors partially provide reliability in their satellite by using five fully redundant processors.   As the IlliniSat project has already fixed its hardware requirements and as we wish to produce advice and solutions that can be directly applied to IlliniSat, we do not consider these approaches in this work.


Looking into commercial CubeSat implementations, we were not able to 
find any apparent software reliability solutions in their systems.
\footnote{\url{http://www.clyde-space.com/cubesat_shop/software}}
\footnote{\url{http://www.isispace.nl/cms/index.php/products-and-services/missions}}
\footnote{\url{http://www.gomspace.com/index.php?p=products-software}}
The suppliers mentions that their system is reliable, but this is because
they are flight tested, not because of any software reliability techniques that
have been employed.

However, the commercial suppliers do offer hardware solutions. Most commonly
the hardware is radiation-hardened to lessen the likelihood of transient cosmic ray
failures in space.\footnote{\url{http://www.spacemicro.com/space-components.html}}
\footnote{\url{http://www.aacmicrotec.com/images/Products/Components/MCC_website.pdf}}




