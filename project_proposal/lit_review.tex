\section{Literature Review}

There is already an existing body of work that deals with CubeSats and
investigations of their reliability.  In \cite{odegaard2013error} the author
discusses developing software systems to increase the reliability of a CubeSat.
One of the systems discussed in \cite{odegaard2013error}, the use of a
recovery mode, are already used in IlliniSat.  \cite{odegaard2013error} also
proposes solutions for protecting the satellite's program code, although its
implementation is unable to handle the case of a corrupted kernel with no saved
system checkpoints; our proposed approach stores multiple copies of the kernel,
so it is able to recover even if a single kernel image is completely corrupted.

There has also been significant work in using hardware-based reliability measures in small satellite designs.  \cite{toorian2008cubesat} mentions redundancy and watchdog timers as two solutions to ensure CubeSat reliability.  In \cite{passerone2008design}, the authors partially provide reliability in their satellite by using five fully redundant processors.   As the IlliniSat project has already fixed its hardware requirements and as we wish to produce advice and solutions that can be directly applied to IlliniSat, we do not consider these approaches in this work.
