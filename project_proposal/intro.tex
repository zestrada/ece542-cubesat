\section{Introduction}
\subsection{IlliniSat}
A CubeSat is a small satellite ($\sim$~1kg) that can be delivered as a secondary
payload or as a ``shared ride'' with a number of other satellites, thus lowering
the cost to launch\cite{toorian2008cubesat}.  With an estimated cost of roughly
\$52,000\footnote{\url{http://www.satmagazine.com/story.php?number=602922274}}
to put an educational CubeSat in orbit, CubeSats are well within the budget of
many universities.  The University of Illinois has a
CubeSat\footnote{\url{http://cubesat.ae.illinois.edu/index.php}} program in
which a class and dedicated team have been developing a scalable picosatellite
($<$~1kg) bus, IlliniSat-2.  IlliniSat-2 is intended to be used for multiple
missions, the first of which (Lower Atmosphere/Ionosphere Coupling Experiment,
LAICE) is planned to launch around December 2014.  This mission will involve
three scientific payloads: one from Illinois and two from Virginia Tech.  In
collaboration with Bindu Jagannatha and Alex Ghosh of the IlliniSat-2 team, this
project aims to analyze and enhance the reliability of the computing systems
used in IlliniSat-2.

The IlliniSat-2 bus contains various systems, and this project focuses on the
Command and Data Handling (C\&DH) system.  The C\&DH system is responsible
for maintaining the mission schedule and coordinating the communication of the
satellite with the ground station.  The C\&DH system is based on a MitySOM-335x
Processor Card, built with a TI AM335x Application Processor System-on-Chip
(SoC).\footnote{\url{http://www.criticallink.com/wp-content/uploads/MitySOM-335x_Datasheet.pdf}}
An embedded Linux distribution, Arago, is run on top of the TI SoC and
IlliniSat's functionality is implemented in set of userspace deamons. The
standard \fix{atd} utility is used to schedule tasks.
%FIXME: include C&DH figure 

Currently, the system has a watchdog timer that will reboot the flight computer
if it does not receive a heartbeat every 60 seconds.  The watchdog is part of a
health monitoring system that also observes the state of the battery, satellite 
attitude, temperature, etc...  If the health monitor determines that the system
is in a faulty state, it will transition C\&DH into a recovery mode (e.g. if the
battery is low, switch off all nonessentials and point the satellite towards the
sun).
%FIXME: do we want the health monitor figure?
%FIXME: define SPoF

\subsection{Design Constraints}
The IlliniSat bus has many hardware and software requirements that constrain our
improvements of the system. Since the IlliniSat team has already chosen their
hardware, we will have to make all of our changes in software. Given the nature
of the mission, the software is required to have mission critical components
that cannot fail in order to keep the satellite working. The main design constraints that will be put on us while improving the system are the following:
\begin{itemize}
  \item Preserve real-time deadlines
  \item Minimize additional power consumption
  \item Avoid complex solutions
\end{itemize}
The IlliniSat satellite needs to be constantly reacting to its environment so that it can communicate to the ground and so that it can monitor the health of the system, among other things. Because of this, strict real-time deadlines are required and cannot be missed. 

Since the IlliniSat is such a small satellite that will also be going on extended missions, power is a major concern of the team. If too much power is consumed, then components may start to overheat and the satellite will need to be charging very often. As well, if the satellite isn't able to charge due to eclipse or bad positioning, the system could die out completely.

Finally, the IlliniSat team has requested that we keep our solution simple so that it can be easily implemented and easily maintained, even by team members that have a small amount of experience in programming and/or reliability. 
