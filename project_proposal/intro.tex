\section{Introduction}
\subsection{IlliniSat}
A CubeSat is a small satellite ($\sim$~1kg) that can be delivered as a secondary
payload or as a ``shared ride'' with a number of other satellites , thus
lowering the cost to launch\cite{toorian2008cubesat}.  With an estimated cost of
roughly \$52,000\footnote{http://www.satmagazine.com/story.php?number=602922274}
to put an educational CubeSat in orbit, CubeSats are well within the budget of
many universities.  The University of Illinois has a
CubeSat\footnote{http://cubesat.ae.illinois.edu/index.php} program in which a
class and dedicated team have been developing a scalable picosatellite ($<$~1kg)
bus, IlliniSat-2.  IlliniSat-2 is intended to be used for multiple missions, the
first of which is planned to launch around December 2014.  This project aims to
analyze and enhance the reliability of the computing systems of the IlliniSat-2
bus.

Previous work has investigated the reliability of CubeSats systems built using
commodity computing devices \cite{odegaard2013error}.
%FIXME: include figure about C&DH
\subsection{Design Constraints}
