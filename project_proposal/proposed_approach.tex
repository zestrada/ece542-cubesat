\section{Proposed Approach}
\subsection{M\"obius simulation of memory corruption}

\subsection{Develop a consistent boot/recovery procedure}
When the project members were conversing with the IlliniSat team, one major
concern that the IlliniSat team had was that the system may not boot up
correctly after the watchdog resets the system (or even after launch).  This can
happen for a number of reasons, but one of the main concerns is corruption of
the Operating System kernel.  If the OS kernel does not boot, this mission will
end in catastrophic failure.  One idea is to address this is store multiple
copies of the OS kernel, along with checksums for the kernel image.  The
bootloader can compute the checksum at boot time and failover to the next kernel
image if it does not match the stored checksum.  While this removes the stored
kernel image as a SPoF, it still leaves the bootloader as a SPoF.  However, the
bootloader's code is significantly smaller than the Linux kernel and therefore
should be less susceptible to bit errors.  If time permits, we can investigate
similar techniques for the bootloader. 

\subsection{Develop a flash patrol read daemon}
In addition to storing the Operating System and programs, IlliniSat uses flash
memory to store up to 30 days of scientific data before it can be transmitted to
the ground station.  To ensure that the data is not corrupted, we plan to
implement a flash patrol read daemon that will sweep over the flash memory and
validate the contents. We plan to do this using a checksum or possibly even an
in memory cache (if storage space permits).  Storing the detection data in
memory allows us to take advantage of the fact that DRAM is more reliable than
flash memory in CubeSats~\cite{odegaard2013error}.  The patrol will run in the
background to make certain that it does not interfere with the real-time
deadlines that are mission critical.  In addition, the patrol will be
implemented such that it checks the flash that is being used instead of just
iterating through its entirety (e.g.  data that has already been transmitted
will no longer be validated). With regards to power consumption, we can ensure
that the daemon only runs while the satellite has sufficient power remaining.
Adding the patrol daemon to the IlliniSat system will help safeguard that no
data is lost on the satellite, allowing its scientific missions to be
successful.

\subsection{Develop a method to increase reliability of ground station
communications}
The ground station is able to send commands from the ground to the IlliniSat satellite. These commands
can change the state of the satellite and also the configuration settings. This makes sense because the
researchers on the ground may want to update the mission objectives or fix new bugs. However, this is
also a lot of power for the ground station. If the commands were somehow changed in transit by cosmic 
rays or communication failure, the satellite could be sent into an unknown state with unknown settings. 
It is then possible that the satellite could keep restarting, hang, or just crash altogether. All of these are
unacceptable failures and must be avoided. So we plan to develop a system of validating the 
communications that are sent from the ground before actually interpreting them. This method could 
involve a checksum or even embedded commands that are stored on the satellite and then indexed into
by the commands sent from the ground. Ensuring reliability of these commands will ensure a stable state
on the satellite and thwart transient errors.

