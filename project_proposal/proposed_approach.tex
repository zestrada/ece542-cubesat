\section{Proposed Approach}
\subsection{M\"obius simulation of memory corruption}

\subsection{Develop a consistent boot/recovery procedure}
When the project members were conversing with the IlliniSat team, one major
concern that the IlliniSat team had was that the system may not boot up
correctly after the watchdog resets the system (or even after launch).  This can
happen for a number of reasons, but one of the main concerns is corruption of
the Operating System kernel.  If the OS kernel does not boot, this mission will
end in catastrophic failure.  One idea is to address this is store multiple
copies of the OS kernel, along with checksums for the kernel image.  The
bootloader can compute the checksum at boot time and failover to the next kernel
image if it does not match the stored checksum.  While this removes the stored
kernel image as a SPoF, it still leaves the bootloader as a SPoF.  However, the
bootloader's code is significantly smaller than the Linux kernel and therefore
should be less susceptible to bit errors.  If time permits, we can investigate
similar techniques for the bootloader. 

\subsection{Develop a flash patrol read daemon}

\subsection{Develop a method to increase reliability of ground station
communications}

