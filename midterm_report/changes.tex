\section{Changes to Goals} 

\subsection{M\"obius Simulation}
We have changed the focus of the M\"obius simulation.  Previously, we intended to create a model to determine the probability that the Flash memory that stored scientific data during a mission would be corrupted.  Now, we are creating a model to determine the impact of our boot and recovery procedures on the system's ability to recover from failures.  We feel that this change will allow us to provide a more quantitative analysis to Task B and will make better use of techniques discussed in class.

\subsection{Consistent Boot/Recovery Procedure}
The plan remains the same for this project, we have just scoped out how exactly
we plan to deliver on this. As a simplified description, IlliniSat uses the
u-boot bootloader to load the Linux kernel.  We decided that we will modify
u-boot to calculate the CRC of multiple kernel images and boot an image that
verifies.  In order to develop this without purchasing a board or slowing down
the rest of IlliniSat's software development, we decided use QEMU with ARM. The
MityARM-335x used in IlliniSat is not available in QEMU and adding it would be
far beyond the scope of the class (though useful for future work in IlliniSat).
Therefore, we chose to use a standard ARM board that is often used with QEMU,
the Versatile PowerBoard.  Though this uses an older processor (ARM9 vs. Cortex
A-8), we believe that the development procedure and codepaths should be similar
enough to be useful for IlliniSat.


\subsection{Flash Patrol Read Daemon}
The plan for the daemon is the same as the initial plan. We are almost finished
with creating and modifying CRC values for every file that is created/modified. 
The other half of that coin will be to check the values that have already been
created for validity. This should be attainable as long as we are able to 
supress any race conditions that this code will add in the code. Also, it is
not noted in the initial proposal, but the CRC creation daemon will use signals 
to intelligently go to sleep and be woken up instead of just chewing up cycles 
and sleeping for a certain period of time. The validity checking daemon won't need
to use signals, since there will always be files to check. It can just validate and sleep
regularly.