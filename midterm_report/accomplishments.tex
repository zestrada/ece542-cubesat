\section{Accomplishments}

\subsection{M\"obius Simulation}

\subsubsection{Failure Model}

\subsubsection{M\"obius Baseline Model}

\subsection{Consistent Boot/Recovery Procedure}
The first challenge with this project was to get a working development
environment off the ground.  Installing QEMU was fairly straightforward, but
cross-compiling and getting u-boot and Linux to work took some effort.  We tried
a number of different ARM boards implemented in QEMU and eventually settled on
the ``versatilepb'' since it was the best documented.\footnote{http://balau82.wordpress.com/2010/03/10/u-boot-for-arm-on-qemu/}$^{,}$\footnote{http://elinux.org/Virtual\_Development\_Board}

After getting u-boot to work, we built a Linux kernel (version 3.2.67 was chosen
as IlliniSat is using a 3.2 based kernel) and booted that both with and without 
u-boot.  To demonstrate that the OS is working properly, we tried to get a
regular user process to run (by specifying it as the \fix{init} process).
However, this was highly non-trivial since it depended on things like system
call linkage, a working libc, and a filesystem.  Since the main project involves
the boot procedure, we will just boot the Linux kernel and let it panic by not
finding an init process. We intend to fix this for a proper demonstration in the
future.

It was discovered the u-boot has a built-in CRC-32 command, which we plan to
utilize. U-boot has two main ways of adding code: via scripts and standalone
applications.  After discovering to enable scripts (by modifying the
configuration header file), we have have decided to use the script method as
that is the most portable option. After properly configuring qemu and u-boot, we
have demonstrated the ability to get a working test script stored on a virtual
flash device.  This script modifies the u-boot environment and next we will work
on building the boot-assurance script.  


\subsection{Flash Patrol Read Daemon}

