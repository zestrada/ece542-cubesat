\section{Task 3}
\subsection{What is the percentage of faulty DRAM?  (i.e., DRAM with permanent
memory faults) }

\subsubsection{How can you spot them in your dataset?}
We attempt to follow industry standard practices and consider any DIMM that
would warrant an in-the-field replacement as ``faulty.''  The two criteria used
are: (1) if the DIMM has any uncorrectable errors (2) if the number of
correctable errors per unit time exceeds a certain threshold.  For criteria (1),
there was only one machine in the dataset that satisfied the requirement.  To
set an appropriate threshold for (2), we again looked at the study in
\cite{schroeder2009dram}.  The authors found that on average, 3351--4530
correctable errors were experienced per year.  Thus, we determined that memory
exhibiting 5000 or more correctable errors per year (scaled to our dataset) is
faulty.  To pinpoint the faulty DRAMs, we follow a similar procedure to task 2
and find out which DRAMs are affected by decoding the symbol.  Since in
Chipkill, each DRAM corresponds to a symbol, we can determine the number of
faulty chips per DIMM.  However, without knowing the memory interleaving
configuration of the DIMMs, we cannot isolate which faults belong to which DIMM.
Here, we make the assumption that only one symbol per DIMM per machine is
faulty.  That is, if we see the same symbol twice for the same node, we assume
it is from the same DIMM. This assumption is likely safe since having two DIMMs
fail on the same symbol on the same node is highly unlikely and it should
average out. We also assume that the system configuration is constant throughout
our dataset (no repairs, etc...).

\subsubsection{How would the MTBF change if ALL of the faulty DRAM was fixed?}
To answer this question, we calculate the memory MTBF with and without faults
associated with faulty DRAMs:
\begin{itemize}
  \item Memory MTBF with faulty DRAMs: 57.94283
  \item Memory MTBF without faulty DRAMs: 184.9117
\end{itemize}
we can see that since faulty DRAMs generate a large number of errors, they can
impact the MTBF significantly - the MTBF without faulty DRAMs is 3.2x longer.

\subsubsection{Why would we still have failures after fixing faulty DRAM?}
We can still have failures after fixing faulty DRAM since faults can be caused
by environmental conditions such as cosmic rays or other forms of radiation
(alpha particles, etc...).

\subsubsection{How effective would the Chipkill be after fixing all the faulty
DRAM? Would we still need Chipkill or ECC would be enough?}
To answer this question, we calculated the percentage of multibit errors due to
faulty DRAMs and found that 49.9\% of multibit errors were attributed to faulty
DRAMs.  This means that roughly half of all multibit errors were experienced in
DRAMs that we classified as ``non-faulty.''  Chipkill is still effective after
all faulty DRAMs are replaced since so many multibit errors would still occur.
