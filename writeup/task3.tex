\section{What is the percentage of faulty DRAM?  (i.e., DRAM with permanent
memory faults) }

\subsection{How can you spot them in your dataset?}
We attempt to follow industry standard practices and consider any DIMM that
would warrant an in-the-field replacement as ``faulty.''  The two criteria used
are: (1) if the DIMM has any uncorrectable errors (2) if the number of
correctable errors per unit time exceeds a certain threshold.  For criteria (1),
there was only one machine in the dataset that satisfied the requirement.  To
set an appropriate threshold for (2), we again looked at the study in
\cite{schroeder2009dram}.  The authors found that on average, 3351--4530
correctable errors were experienced per year.  Thus, we determined that memory
exhibiting 5000 or more correctable errors per year (scaled to our dataset) is
faulty with a rather high degree of confidence.

%assumptions, multiple DIMMs, interleaving unsure - assume that two DIMMs don't
%fail on the same system.  since a rare event, Should not skew data and average out.
